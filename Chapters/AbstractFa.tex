\AbstractFa{
با افزایش محبوبیت تلفن‌های هوشمند، استفاده از ابزارهای مبتنی بر سیستم گفت‌وگو نیز به طرز چشم‌گیری افزایش داشته است. درک زبان طبیعی، بخشی حیاتی از یک سیستم گفت‌وگو است؛ چراکه نقص در آن باعث ایجاد گلوگاه در چرخه‌ی عملکرد سیستم گفت‌وگو می‌شود. تشخیص هدف کاربر و پرکردن جای خالی، دو وظیفه‌ی اصلی درک زبان طبیعی هستند. 
شبکه‌های عصبی بازگشتی به طور گسترده برای بهبود این وظایف مورد بررسی قرار گرفته‌اند، اما دارای ضعف‌های شناخته شده‌ای مانند گرادیان محو شونده و زمان آموزش بالا هستند. به تازگی، ترنسفورمر برای رفع ایرادات مذکور معرفی شده است. از طرف دیگر، تعداد کمی از کارهای پیشین، خروجی مدل‌های زبانی را برای کار مورد نظر تعبیه می‌کنند.
در این پایان‌نامه، مدل \lr{CTran} معرفی می‌شود. \lr{CTran} یک مدل رمزنگار-رمزگشا‌ی مبتنی بر شبکه‌ی عصبی کانولوشنی و ترنسفورمر است که برای دو چالش تشخیص هدف و پرکردن جای خالی طراحی شده است. در رمزنگار \lr{CTran}، از بِرت به عنوان فراهم کننده‌ی تعبیه‌ی اولیه واژه ها استفاده شده است. سپس، یک لایه‌ی کانولوشنی با اندازه‌های هسته‌ی متفاوت استفاده شده، خروجی آن ترانهاده و سپس الحاق شده است. در بخش آخر رمزنگار، خروجی به یک پشته‌ی رمزنگار ترنسفورمر تغذیه شده تا تعبیه‌ی جمله‌ی ورودی تکمیل شود. در \lr{CTran} به منظور تولید خروجی برای هر وظیفه، دو رمزگشای جداگانه معرفی شده، که هردو یک رمزنگار را به طور مشترک استفاده می‌کنند. برای رمزگشای تشخیص هدف، مکانیزم توجه به خود و به دنبال آن از یک لایه‌ی خطی تماماً متصل به کار گرفته شده است. برای رمزگشای پرکردن جای خالی، رمزگشای ترنسفورمر تراز شده معرفی گردیده است. برای تراز کردن رمزگشای ترنسفورمر، از ماتریس قطری استفاده شده است. ماتریس قطری، موقعیت‌های متناظر با برچسب‌های هدف در رمزنگار را، در دسترس قرار داده و سایر موقعیت‌ها را مخفی می‌کند. در انتهای کار، به منظور سنجش صحت عملکرد مدل پیشنهادی، مدل بر روی دو مجموعه داده‌ی \lr{ATIS} و \lr{SNIPS} آزمایش گردید. نتایج آزمایش‌ها نشان می‌دهد که مدل پیشنهادی در تشخیص جای خالی، بر روی هر دو مجموعه داده، عملکرد بهتری از مدل‌های پیشین دارد. علاوه بر این، عملکرد دو استراتژی مدل زبانی به عنوان رمزنگار و مدل زبانی به عنوان تعبیه واژه‌ها، سنجیده شد. نتایج نشان می‌دهد که استراتژی استفاده از مدل زبانی تنها به عنوان تعبیه‌ی واژه‌ها، عملکرد بهتری دارد.
}

\KeywordsFa{
1-شبکه‌ی عصبی عمیق،  2-پردازش زبان طبیعی، 3-درک زبان طبیعی،  4-تشخیص هدف در متن، 5-پر کردن جای خالی در متن
}